ls\documentclass{article}
\usepackage[T1]{fontenc}
\usepackage[utf8]{inputenc}
\usepackage[francais]{babel}
\usepackage{graphicx}
\usepackage{verbatim}
\usepackage{moreverb}
\usepackage{hyperref}
\usepackage{listings}
\usepackage{tikz}

\usetikzlibrary{arrows,positioning,automata,shadows,matrix,calc}

\newcommand{\reporttitle}{Cours OpenCL }     % Titre
\newcommand{\reportauthor}{- Henri Frank \textsc{ANABA}}% Auteur
\newcommand{\reportsubject}{Programmation Multicoeur} % Sujet
\newcommand{\HRule}{\rule{\linewidth}{0.5mm}}
\setlength{\parskip}{1ex} % Espace entre les paragraphes

\hypersetup{
    pdftitle={\reporttitle},%
    pdfauthor={\reportauthor},%
    pdfsubject={\reportsubject},%
    pdfkeywords={rapport} {vos} {mots} {clés}
}

\newcommand{\Aut}{\mathcal{A}}
\newcommand{\N}{\mathbb{N}}
\newcommand{\ra}{\rightarrow}

\begin{document}
% Inspiré de http://en.wikibooks.org/wiki/LaTeX/Title_Creation

\begin{titlepage}

\begin{center}

\begin{minipage}[t]{0.48\textwidth}
  \begin{flushleft}
    \includegraphics [width=50mm]{images/index.png} \\[0.5cm]

  \end{flushleft}
\end{minipage}
\begin{minipage}[t]{0.48\textwidth}
  \begin{flushright}
  %  \includegraphics [width=30mm]{images/logo-societe.jpg} \\[0.5cm]
   % \textsc{\LARGE Entreprise}
  \end{flushright}
\end{minipage} \\[1.5cm]

\textsc{\Large \reportsubject}\\[0.5cm]
\HRule \\[0.4cm]
{\huge \bfseries \reporttitle}\\[0.4cm]
\HRule \\[1.5cm]

\begin{minipage}[t]{0.6\textwidth}
  \begin{flushleft} \large
    \emph{Auteurs :}\\ 
    \reportauthor
  \end{flushleft}
\end{minipage}

\hfill






\begin{minipage}[t]{0.6\textwidth}
  \begin{flushright} \large
    \emph{Responsable :} \\
    Mme.~R \textsc{Namyst}\\
  \end{flushright}
\end{minipage}

\vfill

{\large \today}

\end{center}

\end{titlepage}

\section{Rappel}
\paragraph{}
Parfois lorsqu'on code en openCL il y a des variables qui ne tiennent pas sur un coeur élémentaire. et donc celles ci sont déclarées en mémoire et on prend un surcoût (*2, *3). 

\section{Transposition de matrice}
\paragraph{}
 on passe par une variable temporaire.

\begin{verbatim}
	__Kernel  transpose(__global Float *A,
					.....                 *B)
{
	int x = get_global_id(0);
	     y = get_global_id(1);

	     xloc = get_local_id(0);
	     yloc = get_local_id(1);

	     tmp[xloc][yloc] = A[y*get_global_size(0) + x];
	
	barier (CLK_LOCAL_MEM_FENCE);
	$$$ B[get_global_size(1)*(get_group_id(0)*get_local_size(0))+yloc	
	      + get_group_id(1)	* get_local_size(1) + xloc]=tmp

}
\end{verbatim}

\section{Stencil}
!!!!!!!! chargement de tuile
	tile  "OpenCL Best Practice Guide" 
Il existe QDA aussi similaire à OpenCL

\begin{verbatim}
__kernel stencil2D(A,B)
	B[i][j] = C0 * A[][]
\end{verbatim}

\paragraph{}
En motifiant le programme de la transposée en rajoutant +1 dans la variable tmp
__local Float tm[TILE][TILE1];
le programme va plus vite. 
Car les threads qui était en ligne , il écrive en colonne. sur des bancs physiques différents. 
(ruse pour accélerer le programme).

\section{}
\paragraph{}
\today
\paragraph{}
xeon hi acptuel : Knight Corner(knc)\\
	 		   Knight Landing(knl)\\



\paragraph{}
Des intructions ont été rajoutées  pour palier la diffuclté du openCL.

\end{document}

